\documentclass[11pt, a4paper]{article}

\usepackage[utf8]{inputenc}
\usepackage[T1]{fontenc}
\usepackage{microtype}
\usepackage[francais]{babel}
%\usepackage{xcolor}
\usepackage{amsmath}
\usepackage{amssymb}
% Rappel%Pour faire un symbole comme R pour les réels:
% $\mathbb{R} $

\title{Explication du tirage des gens}
\begin{document}
%Définition de commandes
%\newcommand{\impo}[1]{\emph{#1}}
%Fin de définition de commandes

\maketitle

\section{Première version}
Dans la version actuelle des choses, l'algorithme forme toujours les même groupes, puisqu'il tire au
les personnes dans leur ordre de participation. Il n'y a pas assez de hasard dans le processus (il
suffit le faire tourner et voir l'évolution de la matrice Organisation).

Donc, il faut pour cela «coefficienter» la probabilité de participation par l'inverse de taux de
participation (moins on a participé, plus on a de chance de participer.)

Mon idée était de prendre une fonction décroissante sur l'intervalle $[0,1]$ d'aire égale à 1, par
exemple:
\[
    p(t) = c(\exp(-\lambda(t-1))-1)
    \]
Avec $c$ une constante telle que $\int_0^1 p = 1$. J'ai calculé $c = \frac{1}{\lambda}(\exp(\lambda
-1)) +1$.
Puis, de grouper les gens par leur coefficient de participation (c'est tout simplement le nombre de
fois où ils ont participé). On obtient $N$ groupes. Ensuite, on subdivise l'intervalle $[0, 1]$ en
$N$ sous-intervalle de taille $1/N$ noté $I_j$. Puis, on assigne à chaque sous intervalle la valeur
$\int_{I_j} p(t)$.
Puis, on assigne à chaque personne $i$, qui a pour coefficient de participation $j$ la probabilité de participation:
\[
    P(i)= \frac{\int_{I_j}p(t)}{N_j}
\]
Où $N_j$ est le nombre de personne qui ont un coefficient de participation égale à $j$ (c'est-à-dire
le nombre de personne qui ont participé $j$ fois).

Et finalement les personnes sont tirées avec la loi de probabilité suivante (avec $X$ la variable
aléatoire associée à l'identifiant de la personne tirée)
\[
    \mathbb{P}(X = i)= P(i)
\]

\section{Deuxième version}

On a vu que la loi fixée plus haut est défaillante si jamais il y a un très grand nombre de gens qui
ont un coefficient faible et peu de gens à coefficient fort (à cause de la division par $N_j$ on se
retrouve avec une probabilité quasi nulle de tirer des gens qui n'ont pas participer).

Alors, soit on divise par un coefficient, soit on utilise la méthode exposée ci-dessous.

L'idée (plus simple) est de tirer d'abord au hasard un coefficient de participation, puis de
tirer (avec la loi uniforme cette fois ci) une personne qui a le coefficient de participation tirée
précédemment.

Donc, on reprend l'idée plus haut, toujours avec la fonction $p$.
\[
    p(t) = c(\exp(-\lambda(t-1))-1)
    \]
Avec $c$ une constante telle que $\int_0^1 p = 1$. J'ai calculé $c = \frac{1}{\lambda}(\exp(\lambda
-1)) +1$.

Puis, on associe à chaque coefficient le poids suivant:
\[
    N(i)=\int_{I_i}p
    \]
On tire ensuite au hasard le coefficient $c_j$ qui a pour poids $N_j$.

Puis, on tire une personne parmi la liste des gens qui ont le coefficient tiré au hasard, avec une
équiprobabilité.

Ensuite, il faut recommencer pour tirer une autre personne etc..

\end{document}
